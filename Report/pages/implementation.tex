\section{Chi tiết thực hiện}

\subsection{Cấu trúc chương trình}

\subsubsection{Thư viện sử dụng}
Chương trình sử dụng các thư viện Python sau:
\begin{itemize}
	\item \textbf{pandas}: Xử lý và thao tác dữ liệu dạng DataFrame, đọc tập tin CSV
	\item \textbf{numpy}: Tính toán số học, đại số tuyến tính và các phép toán ma trận
	\item \textbf{matplotlib.pyplot}: Tạo biểu đồ và trực quan hóa dữ liệu
	\item \textbf{seaborn}: Tạo biểu đồ thống kê chuyên sâu và ma trận tương quan
	\item \textbf{os}: Quản lý thư mục, đường dẫn tập tin và hệ thống
\end{itemize}

\subsection{Module xử lý dữ liệu}

\subsubsection{Đọc và xử lý dữ liệu}
\begin{itemize}
	\item Sử dụng \texttt{pd.read\_csv()} để đọc tập tin `p03.train.csv' (9000 mẫu) và `p03.test.csv' (1000 mẫu)
	\item Lưu trữ dữ liệu gốc trong \texttt{train\_raw} và \texttt{test\_raw}
	\item Sử dụng \texttt{drop\_duplicates(keep=`first')} để loại bỏ dữ liệu trùng lặp
	\item Báo cáo chi tiết số lượng dòng ban đầu, số dòng trùng lặp và số dòng còn lại
	\item Lưu dữ liệu đã được làm sạch vào \texttt{train} và \texttt{test}
\end{itemize}

\subsubsection{Phân tích khám phá dữ liệu (EDA)}
Thực hiện phân tích toàn diện trên tập huấn luyện:

\textbf{Thống kê mô tả}:
\begin{itemize}
	\item Sử dụng \texttt{train.info()} để kiểm tra kiểu dữ liệu và missing values
	\item Sử dụng \texttt{train.describe()} để có thống kê tổng quan
	\item Kiểm tra \texttt{train.isnull().sum()} để xác nhận không có giá trị thiếu
\end{itemize}

\textbf{Trực quan hóa dữ liệu} - Tạo 6 biểu đồ chính:
\begin{itemize}
	\item \textbf{Figure 1}: Distribution of Performance Index - Biểu đồ phân phối của biến mục tiêu
	\item \textbf{Figure 2}: Boxplots for numerical features - Biểu đồ hộp cho các đặc trưng số
	\item \textbf{Figure 3}: Distribution of Extracurricular Activities - Phân phối hoạt động ngoại khóa
	\item \textbf{Figure 4}: Correlation Matrix - Ma trận tương quan giữa các đặc trưng
	\item \textbf{Figure 5}: Scatter plots relationships - Biểu đồ phân tán mối quan hệ với Performance Index
	\item \textbf{Figure 6}: Performance comparison by Extracurricular - So sánh hiệu suất theo hoạt động ngoại khóa
\end{itemize}

\subsection{Module cài đặt mô hình chính}

\subsubsection{Lớp \texttt{LinearRegression}}
\textbf{Mục đích}: Cài đặt hoàn chỉnh mô hình hồi quy tuyến tính sử dụng Normal Equation.

\textbf{Thuộc tính}:
\begin{itemize}
	\item \texttt{coefficients}: Mảng các hệ số \(\beta_1, \beta_2, ..., \beta_n\)
	\item \texttt{intercept}: Hệ số chặn \(\beta_0\)
	\item \texttt{feature\_names}: Danh sách tên các đặc trưng
\end{itemize}

\textbf{Phương thức \texttt{fit(X, y)}}:
\begin{enumerate}
	\item Chuyển đổi dữ liệu đầu vào thành numpy arrays
	\item Lưu trữ tên đặc trưng nếu X là DataFrame
	\item Thêm cột bias (cột toàn số 1) cho intercept: \texttt{X\_with\_bias}
	\item Áp dụng Normal Equation: \(\beta = {(X^{T}X)}^{-1}X^{T}y\)
	      \begin{itemize}
		      \item Tính \texttt{XtX = np.dot(X\_with\_bias.T, X\_with\_bias)}
		      \item Tính \texttt{XtX\_inv = np.linalg.inv(XtX)}
		      \item Tính \texttt{Xty = np.dot(X\_with\_bias.T, y)}
		      \item Kết quả: \texttt{beta = np.dot(XtX\_inv, Xty)}
	      \end{itemize}
	\item Xử lý ngoại lệ ma trận suy biến bằng pseudo-inverse
	\item Tách intercept và coefficients từ vector beta
\end{enumerate}

\textbf{Phương thức \texttt{predict(X)}}:
\begin{itemize}
	\item Áp dụng công thức: \(y = \beta_0 + \beta_1x_1 + \beta_2x_2 + \cdots + \beta_{n}x_{n}\)
	\item Sử dụng \texttt{np.dot(X, self.coefficients)} để tính tích vô hướng
	\item Trả về mảng dự đoán
\end{itemize}

\subsubsection{Hàm \texttt{calculate\_mse()}}
\textbf{Mục đích}: Tính Mean Squared Error để đánh giá hiệu suất mô hình.

\textbf{Công thức toán học}:
\[MSE = \frac{1}{n}\sum_{i=1}^{n}{(y_i - \hat{y_i})}^2\]

\textbf{Thực hiện}:
\begin{itemize}
	\item Chuyển đổi \texttt{y\_true} và \texttt{y\_pred} thành numpy arrays
	\item Tính số lượng mẫu: \texttt{n = len(y\_true)}
	\item Áp dụng công thức: \texttt{mse = np.sum((y\_true - y\_pred) ** 2) / n}
\end{itemize}

\subsubsection{Hàm \texttt{k\_fold\_cross\_validation()}}
\textbf{Mục đích}: Thực hiện đánh giá chéo k-fold để đánh giá độ tin cậy mô hình.

\textbf{Tham số}:
\begin{itemize}
	\item \texttt{X}: Ma trận đặc trưng
	\item \texttt{y}: Vector mục tiêu
	\item \texttt{k=5}: Số fold (mặc định 5) - Cân bằng giữa độ chính xác và chi phí tính toán
	\item \texttt{random\_state=42}: Hạt giống ngẫu nhiên (giá trị phổ biến trong Machine Learning) để đảm bảo tái tạo kết quả
\end{itemize}

\textbf{Quy trình thực hiện}:
\begin{enumerate}
	\item Thiết lập hạt giống ngẫu nhiên: \texttt{np.random.seed(random\_state)}
	\item Tạo và xáo trộn chỉ số mẫu: \texttt{np.random.shuffle(indices)}
	\item Chia dữ liệu thành k folds với \texttt{fold\_size = n\_samples // k}
	\item Với mỗi fold i:
	      \begin{itemize}
		      \item Xác định chỉ số test: từ \texttt{i * fold\_size} đến \texttt{(i+1) * fold\_size}
		      \item Chỉ số train: tất cả mẫu còn lại
		      \item Chia dữ liệu theo chỉ số
		      \item Huấn luyện mô hình LinearRegression trên fold train
		      \item Dự đoán trên fold test và tính MSE
	      \end{itemize}
	\item Trả về MSE trung bình: \texttt{np.mean(mse\_scores)}
\end{enumerate}

\subsubsection{Hàm \texttt{display\_model\_formula()}}
\textbf{Mục đích}: Hiển thị công thức hồi quy dưới dạng toán học dễ đọc.

\textbf{Thực hiện}:
\begin{itemize}
	\item Bắt đầu với intercept: \texttt{f"Student Performance = \{model.intercept:.3f\}"}
	\item Duyệt qua từng đặc trưng và hệ số tương ứng
	\item Xử lý dấu "+/-" tự động dựa trên giá trị hệ số
	\item Định dạng: \texttt{f"\{sign\}\{coef:.3f\}×\{feature\}"}
	\item In ra công thức hoàn chỉnh
\end{itemize}

\subsection{Module kỹ thuật đặc trưng}

\subsubsection{Thiết kế 5 mô hình tùy chỉnh}

\textbf{Mô hình 1: Top 2 đặc trưng}
\begin{itemize}
	\item \textbf{Hàm}: \texttt{create\_model1\_features(X)}
	\item \textbf{Đặc trưng}: Hours Studied + Previous Scores
	\item \textbf{Cơ sở}: Dựa trên phân tích tương quan cao nhất từ EDA
	\item \textbf{Thực hiện}: \texttt{return X[[`Hours Studied', `Previous Scores']]}
\end{itemize}

\textbf{Mô hình 2: Top 3 đặc trưng}
\begin{itemize}
	\item \textbf{Hàm}: \texttt{create\_model2\_features(X)}
	\item \textbf{Đặc trưng}: Hours Studied + Previous Scores + Extracurricular Activities
	\item \textbf{Cơ sở}: Mở rộng từ Model 1 với thêm yếu tố hoạt động ngoại khóa
	\item \textbf{Thực hiện}: \texttt{return X[[`Hours Studied', `Previous Scores', `Extracurricular Activities']]}
\end{itemize}

\textbf{Mô hình 3: Đặc trưng đa thức}
\begin{itemize}
	\item \textbf{Hàm}: \texttt{create\_model3\_features(X)}
	\item \textbf{Đặc trưng}: Previous Scores + Previous Scores²
	\item \textbf{Cơ sở}: Nắm bắt mối quan hệ phi tuyến
	\item \textbf{Thực hiện}:
	      \begin{itemize}
		      \item \texttt{X\_new = X[[`Previous Scores']].copy()}
		      \item \texttt{X\_new[`Previous\_Scores\_squared'] = X[`Previous Scores'] ** 2}
	      \end{itemize}
\end{itemize}

\textbf{Mô hình 4: Chuẩn hóa Z-score}
\begin{itemize}
	\item \textbf{Hàm}: \texttt{create\_model4\_features(X)}
	\item \textbf{Đặc trưng}: Hours Studied và Previous Scores được chuẩn hóa
	\item \textbf{Công thức chuẩn hóa}: \(z = \frac{x - \mu}{\sigma}\)
	\item \textbf{Thực hiện}:
	      \begin{itemize}
		      \item Tính mean và std cho mỗi đặc trưng
		      \item \texttt{Hours\_Normalized = (Hours - mean) / std}
		      \item \texttt{Previous\_Normalized = (Previous - mean) / std}
	      \end{itemize}
\end{itemize}

\textbf{Mô hình 5: Đặc trưng tương tác và kỹ thuật}
\begin{itemize}
	\item \textbf{Đặc trưng kỹ thuật:}
	\begin{itemize}
		\item \textbf{Hours\_x\_Previous}: \texttt{X[`Hours Studied'] * X[`Previous Scores']}
		\item \textbf{Study\_Efficiency}: \texttt{X[`Hours Studied'] * X[`Sample Question Papers Practiced']}
		\item \textbf{Total\_Capability}: \texttt{X[`Previous Scores'] + X[`Extracurricular Activities'] * 10}
		\item \textbf{Balance\_Index}: \texttt{X[`Previous Scores'] / (X[`Hours Studied'] + 1)}
	\end{itemize}
	\item \textbf{Ý nghĩa}:
	\begin{itemize}
		\item Tương tác giữa số giờ học và điểm số trước đó
		\item Hiệu quả học tập qua việc luyện tập
		\item Tổng năng lực với bonus từ hoạt động ngoại khóa (nhân 10 để tạo trọng số có ý nghĩa so với Previous Scores)
		\item Chỉ số cân bằng đánh giá hiệu quả (tránh chia 0 bằng cách cộng thêm 1 vào Hours Studied)
	\end{itemize}
\end{itemize}

\subsection{Module đánh giá và báo cáo}

\subsubsection{Yêu cầu 2a: Mô hình 5 đặc trưng}
\textbf{Quy trình thực hiện}:
\begin{enumerate}
	\item Khởi tạo mô hình: \texttt{model\_2a = LinearRegression()}
	\item Huấn luyện: \texttt{model\_2a.fit(X\_train, y\_train)}
	\item Dự đoán: \texttt{y\_pred\_2a = model\_2a.predict(X\_test)}
	\item Đánh giá: \texttt{mse\_2a = calculate\_mse(y\_test, y\_pred\_2a)}
	\item Hiển thị công thức: \texttt{display\_model\_formula(model\_2a, X\_train.columns)}
\end{enumerate}

\subsubsection{Yêu cầu 2b: Đặc trưng tốt nhất}
\textbf{Quy trình so sánh}:
\begin{enumerate}
	\item Định nghĩa danh sách 5 đặc trưng
	\item Với mỗi đặc trưng:
	      \begin{itemize}
		      \item Tạo \texttt{X\_single = X\_train[[feature]]}
		      \item Thực hiện 5-fold cross validation
		      \item Lưu kết quả MSE vào dictionary
	      \end{itemize}
	\item Tìm đặc trưng tốt nhất: \texttt{min(cv\_results, key=cv\_results.get)}
	\item Huấn luyện lại mô hình với đặc trưng tốt nhất
	\item Đánh giá trên tập test
\end{enumerate}

\subsubsection{Yêu cầu 2c: Mô hình tùy chỉnh}
\textbf{Quy trình đánh giá}:
\begin{enumerate}
	\item Định nghĩa dictionary chứa 5 mô hình và hàm tạo đặc trưng
	\item Với mỗi mô hình:
	      \begin{itemize}
		      \item Gọi hàm tạo đặc trưng: \texttt{X\_features = feature\_func(X\_train)}
		      \item Thực hiện 5-fold cross validation
	      \end{itemize}
	\item Lựa chọn mô hình tốt nhất dựa trên MSE thấp nhất
	\item Huấn luyện lại trên toàn bộ tập train
	\item Đánh giá trên tập test
\end{enumerate}